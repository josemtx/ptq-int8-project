La implantación de modelos en dispositivos con recursos limitados motiva técnicas
de compresión y aceleración. La cuantización post-entrenamiento (PTQ) a 8 bits es
una de las más usadas por su simplicidad y por no requerir reentrenamiento.
En esta memoria planteamos un micro-motor de ejecución cuantizada en NumPy y
un problema de optimización: reducir la latencia manteniendo la precisión dentro de un margen.

\textbf{Contribuciones}:
(i) implementación educativa de PTQ con fusión Conv+ReLU;
(ii) formulación de calibración como búsqueda heurística;
(iii) evaluación sistemática con métricas reproducibles.
